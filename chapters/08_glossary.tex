\chapter{Glossaire}
\label{chap:glossaire}

\begin{longtable}{p{0.3\textwidth} p{0.7\textwidth}}
\toprule
\textbf{Terme} & \textbf{Définition} \\
\midrule
\endfirsthead
\toprule
\textbf{Terme} & \textbf{Définition} \\
\midrule
\endhead
\bottomrule
\endfoot

\multicolumn{2}{c}{\textbf{Concepts Généraux}} \\
\midrule
Churn (Perte de Clients) & Phénomène par lequel les clients d'une entreprise cessent d'utiliser ses services. Le taux de churn est un indicateur clé de la satisfaction et de la fidélité des clients. \\
\midrule
Machine Learning (Apprentissage Automatique) & Un domaine de l'intelligence artificielle qui donne aux ordinateurs la capacité d'apprendre à partir de données sans être explicitement programmés. \\
\midrule
IA Explicable (XAI) & Un ensemble de techniques et de méthodes qui permettent de comprendre et d'interpréter les décisions prises par les modèles d'apprentissage automatique. \\
\midrule
Indicateur de Performance Clé (KPI) & Une valeur mesurable qui démontre l'efficacité avec laquelle une entreprise atteint ses objectifs commerciaux clés. \\
\midrule
MLOps (Machine Learning Operations) & Une culture et une pratique qui visent à unifier le développement de systèmes d'apprentissage automatique (Dev) et le déploiement de ces systèmes (Ops). \\
\midrule
\multicolumn{2}{c}{\textbf{Colonnes de l'Ensemble de Données}} \\
\midrule
\texttt{Churn} & La variable cible. Indique si le client a quitté l'entreprise (Oui/Non). \\
\texttt{Contract} & Le type de contrat du client (Mois par mois, Un an, Deux ans). \\
\texttt{Dependents} & Indique si le client a des personnes à charge (Oui/Non). \\
\texttt{DeviceProtection} & Indique si le client a une assurance pour ses appareils (Oui/Non). \\
\texttt{gender} & Le genre du client (Homme/Femme). \\
\texttt{InternetService} & Indique si le client a un service Internet (DSL, Fibre optique, Non). \\
\texttt{MonthlyCharges} & Le montant facturé au client chaque mois. \\
\texttt{MultipleLines} & Indique si le client a plusieurs lignes téléphoniques (Oui/Non). \\
\texttt{OnlineBackup} & Indique si le client a un service de sauvegarde en ligne (Oui/Non). \\
\texttt{OnlineSecurity} & Indique si le client a un service de sécurité en ligne (Oui/Non). \\
\texttt{PaperlessBilling} & Indique si le client utilise la facturation dématérialisée (Oui/Non). \\
\texttt{Partner} & Indique si le client a un partenaire (Oui/Non). \\
\texttt{PaymentMethod} & La méthode de paiement du client. \\
\texttt{PhoneService} & Indique si le client a un service téléphonique (Oui/Non). \\
\texttt{SeniorCitizen} & Indique si le client est une personne âgée (1 pour Oui, 0 pour Non). \\
\texttt{StreamingMovies} & Indique si le client a un service de streaming de films (Oui/Non). \\
\texttt{StreamingTV} & Indique si le client a un service de streaming TV (Oui/Non). \\
\texttt{TechSupport} & Indique si le client a un support technique (Oui/Non). \\
\texttt{tenure} & Le nombre de mois depuis que le client est abonné. \\
\texttt{TotalCharges} & Le montant total facturé au client sur toute la durée de son abonnement. \\
\midrule
\multicolumn{2}{c}{\textbf{Termes Techniques de Modélisation}} \\
\midrule
CatBoost & Un algorithme d'apprentissage automatique basé sur le gradient boosting, optimisé pour la gestion des variables catégorielles. \\
\midrule
One-Hot Encoding & Une technique de prétraitement pour convertir des variables catégorielles en un format numérique que les modèles peuvent comprendre, en créant une colonne binaire pour chaque catégorie. \\
\midrule
Feature Scaling (Mise à l'échelle) & Le processus de normalisation de la plage des variables numériques. \texttt{StandardScaler} est une méthode qui centre les données autour de 0 avec un écart-type de 1. \\
\midrule
Hyperparamètres & Les paramètres de configuration d'un modèle qui ne sont pas appris à partir des données, mais qui sont définis avant le processus d'entraînement (ex: taux d'apprentissage, profondeur des arbres). \\
\midrule
Surapprentissage (Overfitting) & Un problème où un modèle d'apprentissage automatique apprend trop bien les données d'entraînement, au point de mal généraliser à de nouvelles données non vues. \\
\midrule
\multicolumn{2}{c}{\textbf{Métriques d'Évaluation}} \\
\midrule
Accuracy (Exactitude) & La proportion de prédictions correctes parmi le nombre total de cas. \\
\midrule
Precision (Précision) & Parmi toutes les prédictions positives, la proportion de celles qui étaient réellement positives. \\
\midrule
Recall (Rappel) & Parmi tous les cas réellement positifs, la proportion qui a été correctement identifiée par le modèle. \\
\midrule
F1-Score & La moyenne harmonique de la précision et du rappel, fournissant un score unique qui équilibre les deux. \\
\midrule
Courbe ROC et AUC & La courbe ROC (Receiver Operating Characteristic) est un graphique qui illustre la performance d'un classifieur. L'AUC (Area Under the Curve) mesure la capacité globale du modèle à distinguer les classes. Une AUC de 1.0 est parfaite, 0.5 est aléatoire. \\
\midrule
\multicolumn{2}{c}{\textbf{Autres Outils et Bibliothèques}} \\
\midrule
SHAP (SHapley Additive exPlanations) & Une méthode d'IA explicable qui attribue à chaque caractéristique une valeur d'importance pour une prédiction donnée. \\
\midrule
Streamlit & Une bibliothèque Python open-source pour créer et partager des applications web personnalisées pour la science des données. \\
\midrule
Docker & Une plateforme pour développer, expédier et exécuter des applications dans des conteneurs, assurant la cohérence des environnements. \\

\end{longtable}
