\chapter{Annexe}

\section{Code Source Sélectionné}

\subsection{Extrait de app.py (CSS Personnalisé)}
\begin{lstlisting}[language=Python, caption={CSS pour le style du tableau de bord}]
def local_css(file_name):
    with open(file_name) as f:
        st.markdown(f'<style>{f.read()}</style>', unsafe_allow_html=True)

def apply_custom_css():
    css = """
    <style>
        /* --- Global --- */
        @import url('https://fonts.googleapis.com/css2?family=Inter:wght@400;600;700&display=swap');
        
        body {
            font-family: 'Inter', sans-serif;
            color: #2d3748;
            background-color: #f7fafc;
        }
        
        .stApp {
            background-color: #f7fafc;
        }

        /* --- Main Content --- */
        .main .block-container {
            padding-top: 2rem;
            padding-bottom: 2rem;
            padding-left: 5rem;
            padding-right: 5rem;
        }

        /* --- Chart Container --- */
        .chart-container {
            background-color: #ffffff;
            border-radius: 20px;
            padding: 2rem;
            box-shadow: 0 4px 6px -1px rgba(0, 0, 0, 0.1), 0 2px 4px -1px rgba(0, 0, 0, 0.06);
            transition: all 0.3s ease-in-out;
        }
        .chart-container:hover {
            box-shadow: 0 10px 15px -3px rgba(0, 0, 0, 0.1), 0 4px 6px -2px rgba(0, 0, 0, 0.05);
        }
        
        /* ... etc ... */
    </style>
    """
    st.markdown(css, unsafe_allow_html=True)
\end{lstlisting}

\section{Dépendances du Projet}
Le fichier \texttt{requirements.txt} contient la liste des bibliothèques Python nécessaires pour exécuter le projet. L'utilisation d'un environnement virtuel est fortement recommandée pour gérer ces dépendances et éviter les conflits.

\begin{lstlisting}[language=bash, caption={Contenu du fichier requirements.txt}]
pandas>=1.0.0
scikit-learn>=0.24.0
xgboost>=1.0.0
shap>=0.40.0
streamlit>=1.0.0
matplotlib>=3.0.0
seaborn>=0.11.0
joblib>=1.0.0
jupyterlab>=3.0.0
catboost>=1.0.0
streamlit-shap>=1.0.0
\end{lstlisting}

\section{Structure du Projet}
Le code source est organisé de manière modulaire pour faciliter la maintenance et la lisibilité.
\begin{verbatim}
.
|-- data/
|   `-- WA_Fn-UseC_-Telco-Customer-Churn.csv
|-- models/
|   `-- catboost_churn_model.joblib
|-- src/
|   |-- app.py
|   |-- customer_diagnosis_page.py
|   |-- global_analytics_page.py
|   |-- sidebar.py
|   |-- train.py
|   `-- ...
|-- chapters/
|   |-- 01_introduction.tex
|   `-- ...
|-- report.tex
`-- requirements.txt
\end{verbatim}

