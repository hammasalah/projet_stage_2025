\chapter{Analyse Fonctionnelle des Cas d’Utilisation}
\label{chap:analyse_fonctionnelle}

\section{Diagramme de Classes}
Le diagramme de classes ci-dessous modélise les principales entités du système et leurs relations. Il illustre la structure de l'application, depuis l'interface utilisateur jusqu'au modèle de prédiction.

\begin{figure}[H]
    \centering
    \begin{tikzpicture}[
        class/.style={
            rectangle,
            draw=maincolor,
            fill=backcolour,
            rounded corners=3pt,
            thick,
            text width=4.6cm,
            align=left,
            inner sep=6pt
        },
        arrow/.style={->, >=stealth, thick, draw=maincolor}
    ]
        \node[class] (app) at (0,0) {
            \begin{tabular}{@{}l@{}}
                	extbf{StreamlitApp}\\\hline
                	extbf{Attributs}\\sidebar : Sidebar\\\hline
                	extbf{Méthodes}\\run()
            \end{tabular}
        };

        \node[class] (global) at (-6,-4) {
            \begin{tabular}{@{}l@{}}
                	extbf{GlobalAnalyticsPage}\\\hline
                	extbf{Données}\\data : DataFrame\\\hline
                	extbf{Méthodes}\\display\_kpis()\\display\_charts()
            \end{tabular}
        };

        \node[class] (diag) at (6,-4) {
            \begin{tabular}{@{}l@{}}
                	extbf{CustomerDiagnosisPage}\\\hline
                	extbf{Dépendances}\\predictor : ChurnPredictor\\\hline
                	extbf{Méthodes}\\display\_prediction()\\display\_explanation()
            \end{tabular}
        };

        \node[class] (predictor) at (6,-9) {
            \begin{tabular}{@{}l@{}}
                	extbf{ChurnPredictor}\\\hline
                	extbf{Attributs}\\model : CatBoostModel\\explainer : ShapExplainer\\\hline
                	extbf{Méthodes}\\predict(customer)\\explain(customer)
            \end{tabular}
        };

        \node[class] (model) at (2,-13) {
            \begin{tabular}{@{}l@{}}
                	extbf{CatBoostModel}\\\hline
                	extbf{Attributs}\\model\_path : string\\\hline
                	extbf{Méthodes}\\load()\\predict\_proba(data)
            \end{tabular}
        };

        \node[class] (explainer) at (10,-13) {
            \begin{tabular}{@{}l@{}}
                	extbf{ShapExplainer}\\\hline
                	extbf{Attributs}\\explainer : shap.TreeExplainer\\\hline
                	extbf{Méthodes}\\get\_shap\_values(data)
            \end{tabular}
        };

        \node[class] (data) at (-6,-9) {
            \begin{tabular}{@{}l@{}}
                	extbf{CustomerData}\\\hline
                	extbf{Attributs}\\raw\_data : DataFrame\\\hline
                	extbf{Méthodes}\\load\_data()\\preprocess()
            \end{tabular}
        };

        \draw[arrow] (app.south west) .. controls (-2,-2) .. node[near start, left]{composition} (global.north);
        \draw[arrow] (app.south east) .. controls (2,-2) .. node[near start, right]{composition} (diag.north);
        \draw[arrow] (diag.south) -- node[midway, right]{agrégation} (predictor.north);
        \draw[arrow] (predictor.south west) .. controls (4,-11) .. node[midway, left]{utilise} (model.north);
        \draw[arrow] (predictor.south east) .. controls (8,-11) .. node[midway, right]{utilise} (explainer.north);
        \draw[arrow, dashed] (global.south east) .. controls (-4,-7) .. node[midway, left]{dépend de} (data.north);
        \draw[arrow, dashed] (predictor.west) .. controls (0,-9) .. node[midway, below]{dépend de} (data.east);
    \end{tikzpicture}
    \caption{Diagramme de classes du système.}
    \label{fig:class_diagram}
\end{figure}

\textit{Description du diagramme (à compléter avec le diagramme réel) : Le diagramme montre une classe \texttt{StreamlitApp} qui gère l'interface utilisateur. Cette classe est composée de deux pages principales : \texttt{GlobalAnalyticsPage} et \texttt{CustomerDiagnosisPage}. La page de diagnostic interagit avec un \texttt{ChurnPredictor}, qui lui-même encapsule le \texttt{CatBoostModel} et un \texttt{ShapExplainer}. Les données sont représentées par la classe \texttt{CustomerData}.}

\section{Cas d’Utilisation}
Les cas d'utilisation décrivent les interactions entre les acteurs (utilisateurs) et le système pour atteindre un objectif spécifique.

\begin{figure}[H]
    \centering
    \begin{tikzpicture}[
        actor/.style={
            rectangle,
            draw=maincolor,
            rounded corners=2pt,
            thick,
            align=center,
            text width=3.4cm,
            inner sep=4pt,
            fill=backcolour
        },
        usecase/.style={
            ellipse,
            draw=maincolor,
            thick,
            align=center,
            minimum width=3.5cm,
            minimum height=1.2cm,
            fill=backcolour
        },
        connector/.style={-, thick, draw=maincolor}
    ]
        \node[actor] (analyst) at (-5,0) {Analyste m\'etier\\Manager marketing};
        \node[actor] (agent) at (-5,-3.5) {Agent service client\\Responsable de compte};

        \node[usecase] (global) at (1,0) {Analyser le churn global};
        \node[usecase] (diagnosis) at (1,-3.5) {Diagnostiquer un client};

        \node[actor] (app) at (6,-1.75) {Application Streamlit};

        \draw[connector] (analyst) -- (global);
        \draw[connector] (agent) -- (diagnosis);
        \draw[connector] (global) -- (app);
        \draw[connector] (diagnosis) -- (app);
    \end{tikzpicture}
    \caption{Diagramme de cas d'utilisation principal.}
    \label{fig:use_case_diagram}
\end{figure}

\subsection{UC-01 : Analyser le Churn Global}
\begin{itemize}
    \item \textbf{Acteur} : Analyste Métier, Manager Marketing.
    \item \textbf{Description} : L'utilisateur souhaite avoir une vue d'ensemble des tendances de churn au sein de l'entreprise.
    \item \textbf{Scénario Nominal} :
    \begin{enumerate}
        \item L'utilisateur accède à la page "Global Analytics Dashboard".
        \item Le système affiche les KPIs par défaut et les graphiques pour l'ensemble de la base de clients.
        \item L'utilisateur applique un filtre (par exemple, "Contrat = Mois par mois").
        \item Le système met à jour instantanément tous les KPIs et graphiques pour ne refléter que les clients correspondant au filtre.
        \item L'utilisateur analyse les visualisations pour identifier des tendances (par exemple, "le taux de churn est de 45\% pour les contrats mensuels").
    \end{enumerate}
\end{itemize}

\begin{figure}[H]
    \centering
    \begin{tikzpicture}[
        node distance=1.6cm,
        process/.style={
            rectangle,
            draw=maincolor,
            rounded corners=2pt,
            thick,
            align=center,
            text width=5.4cm,
            inner sep=6pt,
            fill=backcolour
        },
        startstop/.style={
            circle,
            draw=maincolor,
            thick,
            minimum size=0.9cm,
            fill=backcolour
        },
        connector/.style={->, >=stealth, thick, draw=maincolor}
    ]
        \node[startstop] (start) {\textbf{Début}};
        \node[process, below of=start] (open) {L'utilisateur ouvre la page Global Analytics Dashboard};
        \node[process, below of=open] (load) {Le système charge les données agrégées et affiche les KPIs par défaut};
        \node[process, below of=load] (filter) {L'utilisateur applique des filtres (contrat, tenure, services, etc.)};
        \node[process, below of=filter] (update) {Recalcul immédiat des indicateurs et graphiques filtrés};
        \node[process, below of=update] (analyse) {Analyse des tendances et identification des segments à risque};
        \node[startstop, below of=analyse] (end) {\textbf{Fin}};

        \draw[connector] (start) -- (open);
        \draw[connector] (open) -- (load);
        \draw[connector] (load) -- (filter);
        \draw[connector] (filter) -- (update);
        \draw[connector] (update) -- (analyse);
        \draw[connector] (analyse) -- (end);
    \end{tikzpicture}
    \caption{Flux du cas d'utilisation ``Analyser le churn global''.}
    \label{fig:use_case_global_flow}
\end{figure}

\subsection{UC-02 : Diagnostiquer un Client Spécifique}
\begin{itemize}
    \item \textbf{Acteur} : Agent du Service Client, Responsable de Compte.
    \item \textbf{Description} : L'utilisateur a besoin de comprendre le risque de churn pour un client particulier et les raisons de ce risque.
    \item \textbf{Scénario Nominal} :
    \begin{enumerate}
        \item L'utilisateur navigue vers la page "Customer Diagnosis".
        \item Il sélectionne un ID client dans le menu déroulant.
        \item Le système :
        \begin{itemize}
            \item Récupère les données du client.
            \item Fait une prédiction de churn en utilisant le modèle CatBoost.
            \item Affiche la probabilité de churn sous forme de jauge.
            \item Calcule les valeurs SHAP pour cette prédiction.
            \item Affiche le graphique "force plot" SHAP, détaillant les facteurs de risque et de rétention.
        \end{itemize}
        \item L'utilisateur analyse le graphique et identifie que le client est à risque à cause de ses frais mensuels élevés.
        \item L'utilisateur peut alors engager une action ciblée (par exemple, contacter le client pour lui proposer une offre promotionnelle).
    \end{enumerate}
\end{itemize}

\begin{figure}[H]
    \centering
    \begin{tikzpicture}[
        node distance=1.6cm,
        process/.style={
            rectangle,
            draw=maincolor,
            rounded corners=2pt,
            thick,
            align=center,
            text width=5.2cm,
            inner sep=6pt,
            fill=backcolour
        },
        startstop/.style={
            circle,
            draw=maincolor,
            thick,
            minimum size=0.9cm,
            fill=backcolour
        },
        connector/.style={->, >=stealth, thick, draw=maincolor}
    ]
        \node[startstop] (start) {\textbf{Début}};
        \node[process, below of=start] (open) {L'utilisateur ouvre la page Customer Diagnosis};
        \node[process, below of=open] (select) {Sélection d'un ID client dans la liste déroulante};
        \node[process, below of=select] (fetch) {Le système charge les données brutes du client};
        \node[process, below of=fetch] (predict) {Prédiction du churn avec CatBoostModel};
        \node[process, below of=predict] (explain) {Calcul des valeurs SHAP et génération des visualisations};
        \node[process, below of=explain] (display) {Affichage de la probabilité de churn et des facteurs clés};
        \node[process, below of=display] (decide) {L'utilisateur planifie une action de rétention ciblée};
        \node[startstop, below of=decide] (end) {\textbf{Fin}};

        \draw[connector] (start) -- (open);
        \draw[connector] (open) -- (select);
        \draw[connector] (select) -- (fetch);
        \draw[connector] (fetch) -- (predict);
        \draw[connector] (predict) -- (explain);
        \draw[connector] (explain) -- (display);
        \draw[connector] (display) -- (decide);
        \draw[connector] (decide) -- (end);
    \end{tikzpicture}
    \caption{Flux du cas d'utilisation ``Diagnostiquer un client spécifique''.}
    \label{fig:use_case_diagnosis_flow}
\end{figure}

\section{Règles métier et validations}
Pour garantir la cohérence du système, plusieurs règles métier encadrent l'exécution des cas d'utilisation :
\begin{itemize}
    \item \textbf{Seuils de churn} : Les alertes ne sont déclenchées que lorsque la probabilité dépasse 60\%, afin d'éviter la saturation des équipes.
    \item \textbf{Historique obligatoire} : Toute action de rétention doit être consignée avant de clore un dossier afin de conserver une traçabilité complète.
    \item \textbf{Mise à jour des données} : Les prédictions ne sont valables que pour une extraction de données datée de moins de 30 jours ; au-delà, une bannière d'avertissement invite à rafraîchir le jeu de données.
\end{itemize}

\section{Correspondance besoins-fonctions}
La Table~\ref{tab:traceabilite_besoins} relie chaque besoin fonctionnel aux éléments de solution implémentés.

\begin{table}[H]
    \centering
    \begin{tabular}{p{3cm} p{5cm} p{5cm}}
        	oprule
        	extbf{Besoin} & \textbf{Fonctionnalité associée} & \textbf{Chapitre de référence} \\
        \midrule
        BF-01 & Tableau de bord global & Chapitre~\ref{chap:fonctionnalites} \newline Section ``Tableau de Bord Analytique Global'' \\
        BF-04 & Diagnostic client & Chapitre~\ref{chap:fonctionnalites} \newline Section ``Outil de Diagnostic Client'' \\
        BF-05 & Prédiction individuelle & Chapitre~\ref{chap:modelisation_evaluation} \newline Section ``Sélection et entraînement du modèle'' \\
        BF-06 & Explication de la prédiction & Chapitres~\ref{chap:modelisation_evaluation} et \ref{chap:analyse_fonctionnelle} \\
        BNF-01 & Optimisation performance & Chapitre~\ref{chap:technologies} \newline Section ``Architecture logicielle'' \\
        BNF-04 & Accessibilité & Chapitre~\ref{chap:interface_utilisateur} \\
        \bottomrule
    \end{tabular}
    \caption{Traçabilité entre besoins et solution implémentée.}
    \label{tab:traceabilite_besoins}
\end{table}
