\chapter{Analyse Fonctionnelle des Cas d’Utilisation}
\label{chap:analyse_fonctionnelle}

\section{Diagramme de Classes}
Le diagramme de classes ci-dessous modélise les principales entités du système et leurs relations. Il illustre la structure de l'application, depuis l'interface utilisateur jusqu'au modèle de prédiction.

\begin{figure}[H]
    \centering
    % Placeholder for Class Diagram
    \fbox{\parbox[c][10cm][c]{15cm}{\centering \Large Diagramme de Classes \\ (à insérer ici)}}
    \caption{Diagramme de classes du système.}
    \label{fig:class_diagram}
\end{figure}

\textit{Description du diagramme (à compléter avec le diagramme réel) : Le diagramme montre une classe \texttt{StreamlitApp} qui gère l'interface utilisateur. Cette classe est composée de deux pages principales : \texttt{GlobalAnalyticsPage} et \texttt{CustomerDiagnosisPage}. La page de diagnostic interagit avec un \texttt{ChurnPredictor}, qui lui-même encapsule le \texttt{CatBoostModel} et un \texttt{ShapExplainer}. Les données sont représentées par la classe \texttt{CustomerData}.}

\section{Cas d’Utilisation}
Les cas d'utilisation décrivent les interactions entre les acteurs (utilisateurs) et le système pour atteindre un objectif spécifique.

\subsection{UC-01 : Analyser le Churn Global}
\begin{itemize}
    \item \textbf{Acteur} : Analyste Métier, Manager Marketing.
    \item \textbf{Description} : L'utilisateur souhaite avoir une vue d'ensemble des tendances de churn au sein de l'entreprise.
    \item \textbf{Scénario Nominal} :
    \begin{enumerate}
        \item L'utilisateur accède à la page "Global Analytics Dashboard".
        \item Le système affiche les KPIs par défaut et les graphiques pour l'ensemble de la base de clients.
        \item L'utilisateur applique un filtre (par exemple, "Contrat = Mois par mois").
        \item Le système met à jour instantanément tous les KPIs et graphiques pour ne refléter que les clients correspondant au filtre.
        \item L'utilisateur analyse les visualisations pour identifier des tendances (par exemple, "le taux de churn est de 45\% pour les contrats mensuels").
    \end{enumerate}
\end{itemize}

\subsection{UC-02 : Diagnostiquer un Client Spécifique}
\begin{itemize}
    \item \textbf{Acteur} : Agent du Service Client, Responsable de Compte.
    \item \textbf{Description} : L'utilisateur a besoin de comprendre le risque de churn pour un client particulier et les raisons de ce risque.
    \item \textbf{Scénario Nominal} :
    \begin{enumerate}
        \item L'utilisateur navigue vers la page "Customer Diagnosis".
        \item Il sélectionne un ID client dans le menu déroulant.
        \item Le système :
        \begin{itemize}
            \item Récupère les données du client.
            \item Fait une prédiction de churn en utilisant le modèle CatBoost.
            \item Affiche la probabilité de churn sous forme de jauge.
            \item Calcule les valeurs SHAP pour cette prédiction.
            \item Affiche le graphique "force plot" SHAP, détaillant les facteurs de risque et de rétention.
        \end{itemize}
        \item L'utilisateur analyse le graphique et identifie que le client est à risque à cause de ses frais mensuels élevés.
        \item L'utilisateur peut alors engager une action ciblée (par exemple, contacter le client pour lui proposer une offre promotionnelle).
    \end{enumerate}
\end{itemize}
