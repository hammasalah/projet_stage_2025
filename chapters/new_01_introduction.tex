\chapter{Introduction}
\label{chap:introduction}

\section{Contexte du projet}
La perte de clients, communément appelée "churn", est un phénomène critique pour toute entreprise, en particulier dans le secteur des télécommunications, qui est hautement compétitif. Le churn représente le pourcentage de clients qui cessent d'utiliser les services d'une entreprise sur une période donnée. Un taux de churn élevé peut avoir des conséquences financières désastreuses, car l'acquisition de nouveaux clients coûte souvent beaucoup plus cher que la fidélisation des clients existants.

Ce projet s'inscrit dans ce contexte et vise à développer une solution complète pour l'analyse et la prédiction du churn client. En utilisant des techniques d'apprentissage automatique (Machine Learning), nous cherchons à identifier les facteurs qui influencent le départ des clients et à construire un modèle prédictif capable d'anticiper quels clients sont les plus susceptibles de résilier leur contrat.

\section{Objectifs}
Les principaux objectifs de ce projet sont les suivants :
\begin{itemize}[label=\textcolor{maincolor}{\textbullet}]
    \item \textbf{Analyser les données clients} : Explorer un ensemble de données de clients de télécommunications pour comprendre les tendances et les caractéristiques des clients qui partent.
    \item \textbf{Identifier les facteurs de churn} : Déterminer les variables clés qui ont le plus d'impact sur la décision d'un client de partir.
    \item \textbf{Construire un modèle prédictif performant} : Développer un modèle de Machine Learning capable de prédire avec une grande précision la probabilité de churn pour chaque client.
    \item \textbf{Assurer l'interprétabilité du modèle} : Utiliser des techniques d'IA explicable (XAI) pour rendre les résultats transparents et exploitables.
    \item \textbf{Développer un tableau de bord interactif} : Créer une application web permettant aux utilisateurs d'explorer les données, de visualiser les analyses et d'obtenir des diagnostics de churn.
\end{itemize}

\section{Impact Métier}
La capacité à prédire le churn avec précision et à en comprendre les causes profondes offre une valeur commerciale considérable. En identifiant les clients à risque avant qu'ils ne partent, une entreprise peut mettre en œuvre des stratégies de rétention proactives et ciblées. Au lieu de campagnes marketing de masse coûteuses, les efforts peuvent être concentrés sur les individus les plus susceptibles de partir, avec des offres personnalisées qui répondent directement aux facteurs de leur insatisfaction.

Ce projet ne se contente pas de fournir une prédiction ; il fournit une explication. Le tableau de bord développé sert de pont entre les data scientists et les équipes opérationnelles. Un responsable marketing peut, en quelques clics, non seulement voir la probabilité de départ d'un client, mais aussi comprendre les raisons sous-jacentes. Cette connaissance permet de transformer une analyse de données complexe en une action client productive et, finalement, en une meilleure fidélisation.

\section{Méthodologie de projet}
Le déroulement du projet suit une approche incrémentale, mêlant exploration de données, sprints techniques et validation métier.
\begin{enumerate}
    \item \textbf{Cadre initial} : Clarification des objectifs et des indicateurs de succès avec les parties prenantes (direction marketing, service client, équipe data).
    \item \textbf{Cycles analytiques} : Chaque itération combine analyses exploratoires, expériences de modélisation et revues d'interprétabilité pour s'assurer de la valeur métier.
    \item \textbf{Intégration applicative} : Les résultats sont industrialisés dans l'application Streamlit à mesure qu'ils sont validés, ce qui favorise une livraison continue.
    \item \textbf{Retours utilisateurs} : Des sessions de démonstration régulières permettent d'ajuster l'ergonomie, les indicateurs affichés et les priorités de développement.
\end{enumerate}

\section{Organisation du rapport}
Le rapport est structuré pour guider le lecteur depuis la compréhension de la problématique jusqu'à la feuille de route future :
\begin{itemize}
    \item Les chapitres \ref{chap:besoins_fonctionnels} à \ref{chap:analyse_donnees} détaillent les besoins et l'état des données.
    \item Les chapitres \ref{chap:modelisation_evaluation} et \ref{chap:fonctionnalites} expliquent respectivement le pipeline de modélisation et la traduction applicative.
    \item Les chapitres \ref{chap:analyse_fonctionnelle} à \ref{chap:interface_utilisateur} décrivent la conception fonctionnelle et l'expérience utilisateur.
    \item Enfin, les chapitres \ref{chap:ameliorations_futures} et \ref{chap:conclusion} ouvrent sur les perspectives et synthétisent les contributions majeures.
\end{itemize}
