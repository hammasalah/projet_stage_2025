\chapter{Conclusion}
\label{chap:conclusion}

Ce projet a permis de concevoir et de développer une solution complète et moderne pour l'un des défis les plus importants du secteur des télécommunications : la prédiction et la compréhension de la perte de clients (churn). En combinant des techniques avancées d'apprentissage automatique avec un fort accent sur l'interprétabilité et une interface utilisateur soignée, nous avons créé un outil qui va au-delà de la simple prédiction.

Le choix du modèle CatBoost s'est avéré judicieux, offrant des performances de prédiction élevées tout en gérant nativement les complexités de nos données. L'intégration de la méthodologie SHAP a été une étape clé, transformant un modèle "boîte noire" en une source d'informations transparente et exploitable. C'est cette capacité à expliquer le "pourquoi" derrière chaque prédiction qui constitue la véritable valeur ajoutée du projet, permettant aux équipes métier de passer de l'analyse à l'action avec confiance.

Le tableau de bord développé avec Streamlit matérialise cette vision en un outil concret. Il réussit à démocratiser l'accès à des analyses complexes, en fournissant une plateforme intuitive où les managers marketing, les analystes et les agents du service client peuvent collaborer pour mettre en œuvre des stratégies de rétention proactives et personnalisées.

En résumé, ce projet a démontré avec succès comment l'intelligence artificielle, lorsqu'elle est conçue de manière centrée sur l'humain, peut devenir un levier stratégique majeur. La solution développée n'est pas seulement un modèle prédictif, mais un véritable système d'aide à la décision, prêt à être intégré dans les processus métier pour réduire le churn, maximiser la satisfaction client et, in fine, améliorer la rentabilité de l'entreprise. Les nombreuses pistes d'amélioration identifiées ouvrent la voie à des développements futurs passionnants qui pourraient encore renforcer l'impact de cet outil.

\section*{Perspectives et engagements}
Les prochaines étapes consisteront à déployer progressivement l'outil auprès de l'ensemble des équipes commerciales et à mesurer rigoureusement son impact sur la rétention. Un comité de gouvernance data sera instauré pour suivre la qualité des données, piloter le ré-entraînement du modèle et garantir le respect des normes éthiques. Enfin, l'ouverture d'API sécurisées permettra d'intégrer la prédiction de churn directement dans les parcours clients digitaux et de créer de nouvelles opportunités de personnalisation.
