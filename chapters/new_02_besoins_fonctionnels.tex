\chapter{Besoins Fonctionnels}
\label{chap:besoins_fonctionnels}

\section{Introduction}
Ce chapitre définit les besoins fonctionnels de l'application de tableau de bord pour l'analyse du churn. L'objectif est de fournir un outil qui non seulement présente des données brutes, mais qui offre également des insights exploitables pour les équipes métier. Les fonctionnalités doivent permettre une exploration intuitive des données et un diagnostic précis des risques de départ des clients.

\section{Liste des Besoins Fonctionnels}
\begin{itemize}
    \item \textbf{BF-01 : Visualisation des Données Globales} : L'application doit présenter une vue d'ensemble des données clients et des indicateurs de churn à travers un tableau de bord principal.
    
    \item \textbf{BF-02 : Filtrage Interactif} : Les utilisateurs doivent pouvoir filtrer les données du tableau de bord global selon plusieurs critères (par exemple, type de contrat, genre, services souscrits) pour affiner leur analyse.
    
    \item \textbf{BF-03 : Affichage des Indicateurs Clés (KPIs)} : Le tableau de bord doit afficher des indicateurs de performance clés (KPIs) de manière claire et visible, tels que le taux de churn global, le nombre total de clients, et l'ancienneté moyenne.
    
    \item \textbf{BF-04 : Diagnostic de Client Individuel} : L'application doit permettre de sélectionner un client spécifique pour analyser son profil en détail.
    
    \item \textbf{BF-05 : Prédiction de Churn Individuel} : Pour un client sélectionné, le système doit calculer et afficher sa probabilité de churn en temps réel, basée sur le modèle pré-entraîné.
    
    \item \textbf{BF-06 : Explication de la Prédiction} : La prédiction de churn pour un client individuel doit être accompagnée d'une explication visuelle (par exemple, un graphique SHAP) qui met en évidence les facteurs contribuant le plus à cette prédiction.
    
    \item \textbf{BF-07 : Navigation Intuitive} : L'application doit être structurée avec une navigation claire, permettant de passer facilement de la vue globale à la vue de diagnostic individuel.
    
    \item \textbf{BF-08 : Interface Utilisateur Esthétique} : L'interface doit être moderne, professionnelle et agréable à utiliser, avec une mise en page et un design soignés.
\end{itemize}

\section{Conclusion}
Ces besoins fonctionnels constituent le cahier des charges pour le développement de l'application. Ils garantissent que le produit final sera un outil puissant et pertinent pour les équipes cherchant à comprendre et à réduire le churn client, en transformant les données brutes en informations stratégiques.
